\documentclass[12pt]{article}
\usepackage[margin=1.25in]{geometry}
\usepackage{titling}
\setlength{\droptitle}{-6\baselineskip}

\usepackage{enumitem}
\usepackage{amsmath}
\usepackage{amssymb}

\title{CSCI 170 -- Homework \#4}
\author{D. Choi}
\date{2022-02-18}


\begin{document}

\maketitle

\paragraph{1.}
{\itshape
    Let $A = \{x : x = 2k \text{ for some integer $k$, and } 0 \leq x < 20\}$, and
    $B = \{x : x = 3^k \text{ for some integer $k$, and } 1 \leq x < 20\}$. Find the cardinality of
    each of the following and show your work.
}
\begin{enumerate}[label=\textbf{\alph*.}]
    \item $A$: $A = \{0, 2, 4, 6, 8, 10, 12, 14, 16, 18\}$; $\boxed{|A| = 10}$
    \item $B$: $B = \{1, 3, 9\}$; $\boxed{|B| = 3}$
    \item $\mathbb{Z} - B$: A bijection can be created between $\mathbb{Z}$ and $\mathbb{Z} - B$.
    Thus, $\boxed{|\mathbb{Z} - B| = |\mathbb{Z}|}$.
    \item $\mathcal{P}(A) \cap \mathcal{P}(A)$:
    $\mathcal{P}(A) \cap \mathcal{P}(A) = \mathcal{P}(A)$;
    $|\mathcal{P}(A)| = 2^{|A|} = 2^{10} = \boxed{1024}$
\end{enumerate}

\paragraph{2.}
{\itshape
    Let the universe $U = \{x : x \text{ is an integer and } 2 \leq x \leq 10\}$. In each of the
    following cases, determine whether $A \subseteq B$, $B \subseteq A$, both, or neither, and
    explain your answer.
}
\begin{enumerate}[label=\textbf{\alph*.}]
    \item $A = \{x : x \text{ is odd}\}$, $B = \{x: x \text{ is a multiple of }3\}$: \\
    \boxed{\text{Neither}}. $1 \in A$, but $1 \notin B$, and $0 \in B$, but $0 \notin A$.
    \item $A = \{x : x \text{ is even}\}$, $B = \{x: x^2 \text{ is even}\}$: \\
    \boxed{\text{Both}}. $x$ is even $\iff$ $x^2$ is even, thus $A = B$.
    \item $A = \{x : x \text{ is even}\}$, $B = \{x: x \text{ is a power of }2\}$: \\
    $\boxed{B \subseteq A}$. All powers of $2$ are even, yet $6$ is even but not a power of $2$.
    \item $A = \{x : 2x + 1 > 7\}$, $B = \{x: x^2 > 20\}$: \\
    \boxed{\text{Neither}}. $4 \in A$, but $4 \notin B$, and $-5 \in B$, but $-5 \notin A$.
    \item $A = \{x : \sqrt{x} \in \mathbb{Z}\}$,
    $B = \{x: x \text{ is a power of }2 \text{ or } 3\}$: \\
    \boxed{\text{Neither}}. $25 \in A$, but $25 \notin B$, and $8 \in B$, but $8 \notin A$.
\end{enumerate}

\pagebreak

\paragraph{3.}
\textit{Prove the following claims.}
\begin{enumerate}[label=\textbf{\alph*.}]
    \item If $A \subseteq B$ and $B \subseteq C$, then $A \subseteq C$: \\
    Assume that $A \subseteq B$ and $B \subseteq C$, but not $A \subseteq C$. Then, there must be
    an element $x$ such that $x \in A$ and $x \notin C$. Because $A \subseteq B$, it must also hold
    that $x \in B$. However, if both $x \in B$ and $x \notin C$, then $B \subseteq C$ cannot be
    true. This is a contradiction. Therefore, if $A \subseteq B$ and $B \subseteq C$, then
    $A \subseteq C$.
    \item If $A \subseteq B$ and $B \subseteq C$, and $C \subseteq A$, then $A = B = C$: \\
    If $A \subseteq B$ and $B \subseteq C$, then $A \subseteq C$. Because $A \subseteq C$ and
    $C \subseteq A$ are both true, $A = C$. By symmetry, $B = A$ and $C = B$. Therefore, because
    equality is transitive, $A = B = C$.
\end{enumerate}

\paragraph{4.}
{\itshape
    For each of the following functions, determine whether it is injective, surjective, both (a
    bijection), or neither. Then, prove or disprove that it is injective, and prove or disprove
    that it is surjective.
}
\begin{enumerate}[label=\textbf{\alph*.}]
    \item $f : \mathbb{R} \mapsto \mathbb{R}$, $f(x) = x^2$: \\
    Not injective. $f(1) = f(-1) = 1$. \\
    Not surjective. There is no $x \in \mathbb{R}$ such that $f(x) = -1$. \\
    Therefore, it is \boxed{\text{neither}}.
    \item $f : \mathbb{N} \mapsto \mathbb{R}$, $f(x) = \sqrt{x}$: \\
    Injective. $f(x) = f(y) \iff \sqrt{x} = \sqrt{y} \iff x = y$. \\
    Not surjective. There is no $x \in \mathbb{N}$ such that $f(x) = -1$. \\
    Therefore, it is \boxed{\text{injective}}.
    \item $f : \mathbb{Z^+} \times \mathbb{Z^+} \mapsto \mathbb{Q^+}$, $f(x, y) = \frac{x}{y}$: \\
    Not injective. $f(1, 2) = f(2, 4) = \frac{1}{2}$. \\
    Surjective. Every rational can be expressed as a ratio of two integers. \\
    Therefore, it is \boxed{\text{surjective}}.
    \item $f : \mathcal{P}(\mathbb{Z}) \mapsto \mathcal{P}(\mathbb{N})$,
    $f(A) = A \cup \mathbb{N}$: \\
    Not injective. $f(\varnothing) = f(\{-1\}) = {\varnothing}$. \\
    Surjective. $\mathbb{N} \subset \mathbb{Z}$. \\
    Therefore, it is \boxed{\text{surjective}}.
    \item $f : \mathcal{P}(\mathbb{Z}) \mapsto \mathcal{P}(\mathbb{Z})$,
    $f(A) = \mathbb{Z} - A$: \\
    Injective and surjective. This function is an involution, and therefore a bijection. \\
    Therefore, it is \boxed{\text{bijective}}.
\end{enumerate}

\end{document}
